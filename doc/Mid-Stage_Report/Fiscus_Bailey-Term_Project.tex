\documentclass[conference]{IEEEtran}

\usepackage{cite}
\usepackage{amsmath,amssymb,amsfonts}
\usepackage{algorithmic}
\usepackage{graphicx}
\usepackage{textcomp}
\usepackage{xcolor}
\def\BibTeX{{\rm B\kern-.05em{\sc i\kern-.025em b}\kern-.08em
    T\kern-.1667em\lower.7ex\hbox{E}\kern-.125emX}}
\begin{document}

\title{Frequency Hopping and Pattern Recognition}

\author{\IEEEauthorblockN{Bailey Fiscus}
\IEEEauthorblockA{\textit{E.C.E. Department of Stevens Institute of Technology} \\
\textit{Stevens Institute of Technology}\\
Hoboken, United States \\
bfiscus@stevens.edu}
}

\maketitle

\begin{abstract}
This will be the abstract.
\end{abstract}

\begin{IEEEkeywords}
wireless, frequency, hopping, pattern, recognition
\end{IEEEkeywords}

\section{Introduction}
This will be the introduction.

\section{Problem Statement}

This will be the problem statement.

%\subsection{This is a subsection}

%This will be the body of the subsection.

\section{Your Solution}
This will be my solution.

\section{Numerical Results and Analysis}
This will be the numerical results and analysis

\section{Conclusions}
This will be the conclusions


\begin{thebibliography}{00}
\bibitem{b1} G. Eason, B. Noble, and I. N. Sneddon, ``On certain integrals of Lipschitz-Hankel type involving products of Bessel functions,'' Phil. Trans. Roy. Soc. London, vol. A247, pp. 529--551, April 1955.

\end{thebibliography}

\end{document}
